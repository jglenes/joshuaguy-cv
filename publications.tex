\begin{rubric}{Publications}
\text{\hangindent=0.5in \textbf{Lenes, J. G.}, Michniewicz, K. A., Chen, J. I., \& Bosson, J. K. (\textit{Under Review}). Gender-atypical mental illness as gender threat. \par }
% \text{\hangindent=0.5in Annis, J., \textbf{Lenes, J. G.}, Westfall, H. A., Criss, A. H., \& Malmberg, K. J.  (\textit{Under Review}). Determining that an empirical phenomenon is insufficient for improving our understanding: A bayesian perspective on the list-length effect in recognition testing. \par }
\text{\hangindent=0.5in Barnes, C. D., Brown, R. P., \textbf{Lenes, J. G.}, Bosson, J. K., \& Carvallo, M. R.  (\textit{in press}). My country, my self: Honor, identity, and defensive responses to national threats, \textit{Self and Identity}. \par }
\text{\hangindent=0.5in Bosson, J. K., Vandello, J. A., Michniewicz, K. S., \& \textbf{Lenes, J. G.} (2012). American men’s and women’s beliefs about gender discrimination: For men, it’s not quite a zero-sum game.  \textit{Journal of Masculinities and Social Change}, \textit{1}(3), 210-239. \par }
\text{\hangindent=0.5in Vescio, T. K., Schlenker, K. A., \& \textbf{Lenes, J. G.} (2010). Power and sexism. In A. Guinote \& T. K. Vescio (Eds.) \textit{The Social Psychology of Power}. New York: The Guilford Press.}
\end{rubric}
